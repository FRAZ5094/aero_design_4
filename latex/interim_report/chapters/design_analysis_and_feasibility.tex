\chapter{Design Analysis and Feasibility}
\section{Safety}
Safety here



\section{Center of gravity (Fraser and Rafał)}

Seeing as the RJ100 is originally a commercial aircraft, converting it to a firefighter aircraft would result in a lot of unnecessary weight, 
for example passenger seats, kitchen and cabin bins ect.
This allows the plane to carry more retardant, meaning better performance for this specific application.
However, by removing the excess weight, the center of gravity is shifted and thus will need to be recalculated.
The RJ100 has an acceptable safe range for the center of gravity in terms of the chord length which is INSERT VALUE to INSERT VALUE.
The plane cannot fly safely if the value of the center of gravity along the x-axis of the plane is outwith this range.
The center of gravity also has to remain within this range before, during and after the ejection of retardant.

\subsection{Method of Analysis}
The process of calculating center of gravity is relatively trivial but can be time consuming if done if done by hand,
especially in this case where there are lots of components being removed, greatly altering the mass.
Using MATLAB instead of hand calculations greatly simplifies the process of changing individual values such as the mass, or the position of objects in the plane, to help achieve the safe center of gravity position. \\ 

According to \cite{baker2020engineering} the formula for the x,y and z value of the center of gravity are shown in the equation \ref{eqn:cog_formula}:

\begin{equation}
\begin{split}
  \bar{x} = \frac{\sum{ \bar{x_{i}} m_{i} }}{ \sum{ m_{i}}} \
  \bar{y} = \frac{\sum{ \bar{y_{i}} m_{i} }}{ \sum{ m_{i}}} \
  \bar{z} = \frac{\sum{ \bar{z_{i}} m_{i} }}{ \sum{ m_{i}}} \
\end{split}
\label{eqn:cog_formula}
\end{equation}

The aircraft mass and center of gravity, before excess weight removal, can effectively be represented in equation \ref{eqn:cog_formula} as a particle with known mass and position.
By representing the aircraft in this way, the center of gravity after the removal of unnecessary weight can be calculated.
The mass of the components to be removed are taken as negative, removing them from the original total mass.

\subsection{Matlab code}
Applying the method above, code was written in MATLAB to to calculate the new cg position through implementing equation \ref{eqn:cog_formula}.
This can be seen below:

\lstinputlisting{../../../code/cg_interia/cg2.m}

The data for the components to be removed were formatted into an excel file named “aircraftitems.xlsx” and then imported to be represented in the variable “Data”.
This data included each component's mass and it's center of gravity position.
These position's were taken relative to the standard aircraft axis with the origin at the nose of the aircraft.

\subsection{Consequence of result of code on design decisions}



\section{Delivery release system / Tank Pressurisation}
Tank pressure here
\section{Structural Support}
Structural Support here
