\chapter{Gravity drop}
\section{Required retardant exit area}
\subsection{Methodology}

%\subsubsection{Required average mass flow rate}
%The goal is to release all the retardant in the tank in 8 seconds.
%Knowing the mass of retardant and the time required to release the retardant the average mass flow rate over the time $\overline{\dot{m}}$ can be calculated:

%\begin{equation}
%\label{eqn:massflow}
%\begin{split}
 %\overline{\dot{m}} = \frac{m}{t}
%\end{split}
%\end{equation}
%where: \\
%\indent m = mass of retardant (kg) \\ 
%\indent t = time retardant needs to be released in (s)

\subsubsection{Instantaneous mass flow rate}
The Instantaneous mass flow rate can be found using this equation:

\begin{equation}
\label{eqn:inst_massflow}
\begin{split}
  \dot{m} = \rho v A 
\end{split}
\end{equation}
where: \\
\indent $\rho$ = density of retardant ($kg/m^{3}$) \\ 
\indent $v$ = velocity of retardant ($m/s$) \\ 
\indent $A$ = area of escape of retardant ($m/s$) \\ 

\subsubsection{Finding the required exit area}
\noindent Using equation \ref{eqn:inst_massflow} for a given $A$ we can find the mass flow at each time step using the velocity at that time step.
This can be integrated with respect to time to find the total mass that flowed through the exit area over the duration.
The mass flow increases linearly over the duration, because the velocity is linear. This means the integration can be done using the area of the triangle under the graph. \\ 

\noindent Using Matlab the total mass that flows though the exit can be calculated for a range of values of A.
Then the smallest value of A that can reach the required total mass flow can be found.




\subsection{Matlab}
\subsubsection{Code}
The Matlab code to calculate the required exit area for volumes of 3000, 4000 and 5000 gallons is shown below:
\lstinputlisting{tank_calculations.m}

\subsubsection{Output of code}

3000 gallons of retardant requires 0.043 m2 area \\
4000 gallons of retardant requires 0.058 m2 area \\
5000 gallons of retardant requires 0.072 m2 area \\
>>
